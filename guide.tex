%!TEX TS-program = xelatex
%!TEX encoding = UTF-8 Unicode

\documentclass[oneside,final]{article}
\usepackage[margin=2.5cm]{geometry}
\usepackage{ctex}
\usepackage{xltxtra}            % to use \XeLeTeX

% hyperlink
\usepackage[colorlinks=true,linkcolor=blue]{hyperref}

% load cjkfonts and set Source Han Sans SC as the default CJK fonts:
%\usepackage[default,mdseries=Light,bfseries=Medium]{cjkfonts}
\usepackage[default,mdseries=Light,bfseries=Medium]{cjkfonts}
% Set Roboto and Source Code Pro, which are installed with TexLive,
% for western fonts:
% localfonts.tex

\newcommand{\fontdir}[0]{/usr/local/texlive/2015/texmf-dist/fonts/}
\newcommand{\robotodir}[0]{\fontdir/truetype/google/roboto/}
\newcommand{\sourcecodeprodir}[0]{\fontdir/opentype/adobe/sourcecodepro/}

\newfontfamily\RobotoThin{Roboto}[
  Extension=.ttf,
  Path=\robotodir,
  UprightFont=*-Thin]

\newfontfamily\RobotoLight{Roboto}[
  Extension=.ttf,
  Path=\robotodir,
  UprightFont=*-Light]

\newfontfamily\RobotoRegular{Roboto}[
  Extension=.ttf,
  Path=\robotodir,
  UprightFont=*-Regular]

\newfontfamily\RobotoMedium{Roboto}[
  Extension=.ttf,
  Path=\robotodir,
  UprightFont=*-Medium]

\newfontfamily\RobotoBold{Roboto}[
  Extension=.ttf,
  Path=\robotodir,
  UprightFont=*-Bold]

\newfontfamily\RobotoBlack{Roboto}[
  Extension=.ttf,
  Path=\robotodir,
  UprightFont=*-Black]

\setmainfont{Roboto}[
  Extension=.ttf,
  Path=\robotodir,
  UprightFont=*-Light,
  BoldFont=*-Medium,
  ItalicFont=*-LightItalic,
  BoldItalicFont=*-MediumItalic]

\setsansfont{Roboto}[
  Extension=.ttf,
  Path=\robotodir,
  UprightFont=*-Light,
  BoldFont=*-Medium,
  ItalicFont=*-LightItalic,
  BoldItalicFont=*-MediumItalic]

\setmonofont{SourceCodePro}[
  Extension=.otf,
  Path=\sourcecodeprodir,
  UprightFont=*-Light,
  BoldFont=*-Semibold,
  ItalicFont=*-LightIt,
  BoldItalicFont=*-SemiboldIt]

\newcommand{\mixthin}[0]{\RobotoThin\NotoSansSCThin}
\newcommand{\mixlight}[0]{\RobotoLight\NotoSansSCLight}
\newcommand{\mixregular}[0]{\RobotoRegular\NotoSansSCRegular}
\newcommand{\mixmedium}[0]{\RobotoMedium\NotoSansSCMedium}
\newcommand{\mixbold}[0]{\RobotoBold\NotoSansSCBold}
\newcommand{\mixblack}[0]{\RobotoBlack\NotoSansSCBlack}

% for code example
\usepackage{listings}
\usepackage{color}

% for better table
\usepackage{booktabs}

\definecolor{mygreen}{rgb}{0,0.6,0}
\definecolor{mygray}{rgb}{0.5,0.5,0.5}
\definecolor{mymauve}{rgb}{0.58,0,0.82}

\lstset{ %
  backgroundcolor=\color{white},   % choose the background color; you must add \usepackage{color} or \usepackage{xcolor}
  basicstyle=\footnotesize\ttfamily,        % the size of the fonts that are used for the code
  breakatwhitespace=false,         % sets if automatic breaks should only happen at whitespace
  breaklines=true,                 % sets automatic line breaking
  captionpos=b,                    % sets the caption-position to bottom
  commentstyle=\color{mygreen},    % comment style
%  deletekeywords={...},            % if you want to delete keywords from the given language
  escapeinside={\%*}{*)},          % if you want to add LaTeX within your code
  extendedchars=true,              % lets you use non-ASCII characters; for 8-bits encodings only, does not work with UTF-8
  frame=single,                    % adds a frame around the code
  keepspaces=true,                 % keeps spaces in text, useful for keeping indentation of code (possibly needs columns=flexible)
%  keywordstyle=\color{blue},       % keyword style
%  language=Octave,                 % the language of the code
%  otherkeywords={*,...},           % if you want to add more keywords to the set
  numbers=left,                    % where to put the line-numbers; possible values are (none, left, right)
  numbersep=5pt,                   % how far the line-numbers are from the code
  numberstyle=\tiny\color{mygray}, % the style that is used for the line-numbers
  rulecolor=\color{black},         % if not set, the frame-color may be changed on line-breaks within not-black text (e.g. comments (green here))
  showspaces=false,                % show spaces everywhere adding particular underscores; it overrides 'showstringspaces'
  showstringspaces=false,          % underline spaces within strings only
  showtabs=false,                  % show tabs within strings adding particular underscores
  stepnumber=1,                    % the step between two line-numbers. If it's 1, each line will be numbered
  stringstyle=\color{mymauve},     % string literal style
  tabsize=2                        % sets default tabsize to 2 spaces
%  title=\lstname                   % show the filename of files included with \lstinputlisting; also try caption instead of title
}

\linespread{1.5}

\begin{document}

\title{cjkfonts 宏包}
\author{Freeman Zhang}
\date{\today{} v0.1}

\maketitle

\section{简介}

\href{https://github.com/zhanggyb/cjkfonts}{cjkfonts} 是一个简单的\XeLaTeX{}宏包,
可用于排版包含中日韩(CJK)文字的文档。其将
\href{https://www.google.com/get/noto/#sans-hans}{Google Noto Sans CJK SC},即
\href{https://github.com/adobe-fonts/source-han-sans}{思源黑体},设置为默认CJK字
体,排版效果如下:

\begin{center}
  Normal Text {\NotoSansSC 正常文本} \\
  \vspace{1em}
  \textbf{Bold Text {\NotoSansSC 粗体文本}} \\
  \vspace{1em}
  \textit{Italic Text {\NotoSansSC 斜体文本}} \\
  \vspace{1em}
  \textbf{\textit{BoldItalic Text {\NotoSansSC 粗斜体文本}}}
\end{center}

\section{使用}

\subsection{签出源码}

\begin{lstlisting}[language=sh]
  $ git clone https://github.com/zhanggyb/cjkfonts
\end{lstlisting}

\subsection{准备字体文件}

运行bootstrap.sh脚本来自动将需要的字体文件下载到fonts目录,这一脚本依赖curl。

\begin{lstlisting}[language=sh]
  $ ./bootstrap.sh
\end{lstlisting}

也可以自行下载并放在 fonts 目录。

\subsection{复制文件}

直接将cjkfonts.sty以及fonts目录复制到你的文档目录。

\subsection{使用宏包}

与其它\LaTeX{}宏包一样,引入cjkfonts宏包只需要在导言区使用:

\begin{lstlisting}[language=TeX]
\usepackage{cjkfonts}
\end{lstlisting}

使用path选项可以选择其它字体所在目录:

\begin{lstlisting}[language=TeX]
\usepackage[path=<mypath>]{cjkfonts}
\end{lstlisting}

使用default选项:

\begin{lstlisting}[language=TeX]
\usepackage[default]{cjkfonts}
\end{lstlisting}

在加载宏包时设置其为默认的CJK字体:

\begin{itemize}
\item 设置CJKmainfont为NotoSans
\item 设置CJKsansfont为NotoSans
\item 设置CJKmonofont为NotoSansMono
\end{itemize}

使用mdseries和bfseries选项可以为常规体和粗体选择不同的字重,例如,以下选项可以让
文字显得更纤细一些:

\begin{lstlisting}
\usepackage[mdseries=Light,bfseries=Medium]{cjkfonts}
\end{lstlisting}

\subsection{示例}

这里是一个简单的示例:

\begin{lstlisting}[language=TeX]
%!TEX TS-program = xelatex
%!TEX encoding = UTF-8 Unicode

\documentclass{article}
\usepackage[default,mdseries=Light,bfseries=Medium]{cjkfonts}

\begin{document}
\begin{center}
  Normal Text 正常文本 \\
  \vspace{1em}
  \textbf{Bold Text 粗体文本} \\
  \vspace{1em}
  \textit{Italic Text 斜体文本} \\
  \vspace{1em}
  \textbf{\textit{BoldItalic Text 粗斜体文本}}
\end{center}
\end{document}
\end{lstlisting}

\textbf{注意:}需要使用\XeLaTeX{}排版引擎来编译源码。

\section{手册}

\subsection{选项}

\begin{center}
\begin{tabular}{ c l }
  \textbf{path=<font dir>}   & 设置Noto Sans CJK SC字体文件路径为<font dir>指定的路径              \\
  \textbf{default}           & 设置默认CJK字体为NotoSansSC                                       \\
  \textbf{mdseries=<weight>} & 设置常规字体字重,<weight>只能从以下表格选择一个,默认为\emph{Regular} \\
  \textbf{bfseries=<weight>} & 设置粗字体字重,<weight>只能从以下表格选择一个,默认为\emph{Bold}      \\
\end{tabular}
\end{center}

\begin{center}
\begin{tabular}{ c c }
\toprule
weight 可取值 & 样例 \\
\midrule
Thin      & {\NotoSansSCThin 朝辞白帝彩云间,千里江陵一日还。}      \\
Light     & {\NotoSansSCLight 朝辞白帝彩云间,千里江陵一日还。}     \\
DemiLight & {\NotoSansSCDemiLight 朝辞白帝彩云间,千里江陵一日还。} \\
Regular   & {\NotoSansSCRegular 朝辞白帝彩云间,千里江陵一日还。}   \\
Medium    & {\NotoSansSCMedium 朝辞白帝彩云间,千里江陵一日还。}    \\
Bold      & {\NotoSansSCBold 朝辞白帝彩云间,千里江陵一日还。}      \\
Black     & {\NotoSansSCBlack 朝辞白帝彩云间,千里江陵一日还。}     \\
\bottomrule
\end{tabular}
\end{center}

\subsection{字体族}

使用同名的字体族来选择字体:

\begin{itemize}
\item {\verb!\NotoSansSC!}
\item {\verb!\NotoSansSCThin!}
\item {\verb!\NotoSansSCLight!}
\item {\verb!\NotoSansSCSemiLight!}
\item {\verb!\NotoSansSCRegular!}
\item {\verb!\NotoSansSCMedium!}
\item {\verb!\NotoSansSCBold!}
\item {\verb!\NotoSansSCBlack!}
\item {\verb!\NotoSansMonoSC!}
\item {\verb!\NotoSansMonoSCRegular!}
\item {\verb!\NotoSansMonoSCBold!}
\end{itemize}

\subsection{CJK支持}

Noto Sans CJK SC已经支持其它繁体中文、韩文和日文字体。

例如:

\begin{center}
  \begin{tabular}{ r l }
    \textbf{简体中文} & 每个人生来平等,享有相同的地位和权利。 \\
    \textbf{繁體中文} & 每個人生來平等,享有相同的地位和權利。 \\
    \textbf{한국의} & 모두가 동일한 태어나 같은 지위와 권리 를 가지고 있다。\\
    \textbf{日本語} & すべての人間は自由であり、かつ、尊厳と権利とについて平等である。\\
  \end{tabular}
\end{center}

\section{已知问题}

\begin{enumerate}
\item bootstrap.sh 脚本只在OS X和Linux系统上测试过。
\end{enumerate}

\section{与Google Roboto字体混合使用}

Noto Sans CJK与
Google Roboto英文字体配合使用非常和谐:

\begin{center}
  {\mixthin Roboto Thin与思源黑体Extra Light混排效果示例}\\
  {\mixlight Roboto Light与思源黑体Light混排效果示例}\\
  {\mixregular Roboto Regular与思源黑体Regular混排效果示例}\\
  {\mixmedium Roboto Medium与思源黑体Medium混排效果示例}\\
  {\mixbold Roboto Bold与思源黑体Bold混排效果示例}\\
  {\mixblack Roboto Black与思源黑体Heavy混排效果示例}
\end{center}

\section{代码实现}

cjkfonts的程序代码非常简单,只是利用
\href{https://github.com/ctex-org/ctex-kit}{xeCJK} 宏包设置排版字体:

\lstinputlisting[language=TeX]{cjkfonts.sty}

\end{document}
